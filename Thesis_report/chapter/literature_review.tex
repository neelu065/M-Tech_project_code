\chapter{Literature review}
\label{literature}

\section{Introduction}
The primary constraint in Aerodynamics Shape Optimization (ASO) is objective function evaluations, which are computationally expensive. This requires optimized methods to reduce the objective function evaluations. Apart from this, the other concern is to parameterize the wing and the choice of design variables. ADODG has suggested a problem (case 6) that involves investigating the existence of multimodality in NACA0012 wing subjected to geometric and aerodynamics constraints. Subsequent sections summarizes the work involving the parameterization of an object's shape and the algorithms to find multi-modal optima within the given design space.

\section{Historical work}

Chernukhin \cite{oleg} investigated the existence of multimodality in aerodynamic design space. The author proposed two novel optimization algorithms that can be used to find the existence of multimodality. They are gradient-based multi-start optimization (GB-MS) based on Sobel sampling and the genetic algorithm. These algorithms are initially tested on the 2-D ASO problem. Using the gradient-based method, the optimizer took 64000 functions evaluation to converge to single optima. However, carry out these many function evaluations would require a lot of computational time. Also, the design space chosen appears to be huge.

Further, the author extends the work to 3-D ASO (similar to that of ADODG case 6), with both transonic and subsonic conditions. The volume mesh is made of 12 blocks with 1.1m nodes. The objective function is to minimize the drag at a Mach number of 0.8 and lift constrained 0.2625. The wing is parameterized using a 2-patch B-spline surface with an initial NACA 0012 baseline wing having a rectangular crossover and a semi span of two. With 125 design variables, the GB optimizer took 60 function evaluations. A similar approach is carried out with GB-MS method involving 128 initial geometries resulting in single optima. In subsonic conditions, that is with Mach number 0.5 and angle of attack between -3 and +6 degree, resulted in seven unique local optima whose objective values differ by more than 5\%. 

The author confirms that 2-D airfoil optimization is unimodal, whereas 3-D wing optimization in subsonic flow conditions is multimodal \cite{oleg:phd}. However, with additional geometric constraints may further reduce multimodality. This work uses the GB-MS algorithm, and gradient-based methods are generally sensitive to the initial population. Also, the use of a gradient-free algorithm in large design variables is computationally expensive and unacceptable. However, with reduced design variables, it is possible to use a gradient-free method efficiently.


In work published by D.J. Poole \cite{Poole1} $et$ $al$., the author addressed the ADODG case 6 problem. The problem is to minimize the drag subjected to aerodynamics and geometric constraints. The wing is of rectangular section with a span length of 3 units and round winglet of 0.06 units. Also, the wing is subjected to compressible, inviscid flow with a Mach number of 0.5 and a constrained lift coefficient of 0.2625. The volume mesh is made of structured multiblock with the convective term being evaluated using Jameson-Schmidt-Turkel (JST) scheme. Nearly 5.5m node points are generated in an eight-block structured C-mesh topology. Further, the surface mesh is made of 129 $\times$ 41 mesh points. The author recommends using radial basis functions (RBFs) for surface control and mesh deformations.

The wing is parameterized using five equally spaced layers along the span direction comprising of 24 control points each along the chord direction. In total, the problem contains 22 design variables with thickness,  z-position, x-position, and a chord at five equally spaced wing sections, resulting in 20 design variables. The remaining two design variables are AoA and Twist [zero at the root and twist angle at the tip]. The AoA is varied within limits set by constraints. The author utilized a hybrid optimizer consisting of particle swarm optimization (PSO) and gravitational search algorithm (GSA). 

The work contains three cases; thickness optimization only, chord optimization only, and full optimization. However, the algorithm which is adopted here result in global optimization. Achieving multimodality is not possible with these algorithms. It is found that the design space chosen for full optimization is multimodal. A strong coupling exists between the chord and thickness of the wing, which increases the multimodality of the problem. However, the existence of multimodality is confirmed by running the optimizer several times. This allows the researcher to rethink on optimizer, which proves multimodality in a single run.

It can be concluded that for any ASO problems, the crucial steps involved are a choice of parameterization technique, selection of optimizer, and the optional step involving dimensional reduction method. Upcoming sections cover the historical work which involve the parameterization techniques, and selection of optimizer.

In the past, attempts have been made to simplify the geometry representation. Meaux $et$  $al$. \cite{Meaux} used (Non-Uniform Rational B-Spline) NURBS to optimize complex 3D geometries. Nadarajah\cite{Nadarajah} compares three different \textbf{parameterization schemes}: B-splines, CST method, and Mesh-point for airfoil geometry. The Mesh-point method involves representing the wing shape with x, y, and z coordinates of grid points. This method would be inefficient as it takes mesh points coordinates as design variables. Since this method involves large matrix operation, evaluating the gradient is difficult, which limits the use of descent algorithm. In the B-spline method, the degree of polynomial representing the surface act as a design variable. With a lower degree to polynomial, it is possible to reduce the dimension of the optimization problem.

The CST method provides the mathematical description of the geometry through a combination of a shape function and class function. The class function provides for a wide variety of geometries. The shape function replaces the complex non-analytic function with a simple analytic function. The simple analytical function uses only a scalable parameter to represent a large domain of design space for the aerodynamic problem. The advantage of CST is, it is efficient in terms of a low number of design variables and allows the use of industrial-related design parameters like the radius of the leading edge or maximum thickness and location\cite{Nadarajah}. The author has shown that the B-spline method allows fewer design variables to represent the wing surface. Due to its low number of design variables, gradient-free methods can be implemented. The complexity involved in the CST method prevents the researcher from using it. 

Gradient-free methods are generally adopted for unconstrained optimization problems. Additional steps need to be followed to implement constraints. One of them is constraint handling as proposed by Deb and Saha \cite{Deb}. The author D.J Poole \cite{Poole2} refers to the constraint handling technique proposed by Deb. Additionally, the author introduces the parallel decomposition of the niching DE algorithm tested on benchmark functions and are reliable.

The author D.J Poole \cite{Poole3} introduced a total of 18 benchmark functions to test the DE-based niching method. Among them, nine are from literature, and remaining are from his work. Along with benchmark functions, different niching methods are also published. The author has shown that, niching methods based on \textbf{neighbourhood search} show better results in terms of quick convergence and reliability of stable niches. However, these algorithms require an extensive amount of cost function evaluations. Hence some modification is necessary to implement them on ASO.

\section{Conclusion}
After analyzing over a few historical work, it can be said that further investigation over finding multimodality is necessary. The work presented by Chernukhin involves parameterization of the wing using B-spline surfaces. However, parameterization techniques called free-form deformation (FFD) can be adopted. The FFD is easy to implement. In case if the design variable appears more in number, then reduced-order modelling methods can be implemented, which will reduce the dimension of the problem by a significant amount.

This work focuses on using the niching algorithms to obtain all possible optima. The niching techniques balance the population diversity by modifying selection, mutation, or crossover stages of Canonical DE. Since they construct on DE, the convergence rate is slow. When combined with these algorithms, the GB method will make them become fast, resulting in Hybrid optimization. The solution is expected to converge faster. The DE based parallel niching algorithms provide a better solution to these class of problems. The scale factor $F$ and cross-factor $ CR $ have a significant impact on the algorithm's performance, such as the quality of the optimal value and convergence rate. There is still no ideal way to determine parameters\cite{chen}. The quality of mesh that is used in D.J.Poole work is super fine. However, with the available computational power, following such a high-quality grid seems complicated. 

According to Chernukhin \cite{oleg}, surrogate modelling is challenging to handle when the number of design variables becomes large. The 3D wing optimization with a viscous model not covered in the paper. However, this involves vast resources to optimize. With the available resources, it becomes difficult to handle. Little work has taken place in optimizing the wing with pre-defined winglet shape. The niching techniques involve a considerable quantity of cost function evaluations. So there needs to reduce cost function evaluation by modifying the code. In the end, solving the ADODG case 6 problems with FFD parameterization, combined with reduced-order modelling and the niching algorithms would lead to a new dimension to approach this problem.