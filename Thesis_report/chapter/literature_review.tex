\chapter{Literature review}
\label{literature}

\section{Introduction}
The main focus when it comes to Aerodynamics Shape Optimization is objective function evaluations, which is CFD simulation-computationally expensive. This result in finding the solution to reduce the objective function evaluations. Apart from this, the other concern is to parameterize the wing and the choice of design variables. ADODG has suggested a problem that involves investigating the existence of multimodality in NACA0012 wing subjected to geometric and aerodynamics constraints. This thesis mainly concentrates on addressing the above problem. The subsequent section addresses the work carried out in the past. It concentrates on papers addressing the parameterization of an object's shape and papers addressing algorithms to find multi-modal optima in the given design space.

\section{Historical work}
In the past, attempts have made to simplify the geometry representation. Meaux $et$  $al$. \cite{Meaux} used (Non-Uniform Rational B-Spline) NURBS to optimize complex 3D geometries. Nadarajah\cite{Nadarajah} compare three different \textbf{parameterization schemes}: B-splines, CST method, and Mesh-point. First, the Mesh-point method involves representing the wing shape with x, y, and z coordinates of grid points. This method would be inefficient as it takes control points coordinates as design variables, and limits the use of descent algorithm since it involves matrix operation. Second, the B-spline method has better control over the design variables by limiting the degree of the polynomial representing the surface. Third, the CST method provides the mathematical description of the geometry through a combination of a shape function and class function. The class function provides for a wide variety of geometries. The shape function replaces the complex non-analytic function with a simple analytic function. The simple analytical function uses only a scalable parameter to represent a large domain of design space for the aerodynamic problem. The advantage of CST is, it is efficient in terms of a low number of design variables and allows the use of industrial-related design parameters like the radius of the leading edge or maximum thickness and location\cite{Nadarajah}. From results, the B-spline method allows fewer design variables to represent the wing surface resulting in low computational cost and a quicker convergence rate. 


In work published by D.J. Poole \cite{Poole1} at el., they use compressible Euler Equation. Multi-block meshing carried out around the NACA 0012 wing. The wing parameterized with five equally spaced rows along span-wise direction comprising 24 control points each along the chord-wise direction. The problem's definition consists of 22 design variables. It comprises thickness, z position, x position, and a chord at five equal spaced wing sections measured along the span-wise direction, which results in 20 design variables. The remaining two variables are AoA and Twist [zero at the root and twist angle at the tip]. They vary AoA with the limits set by constraints. The author utilized the hybrid optimizer consisting of particle swarm optimization (PSO) and gravitational search algorithm (GSA). Thus author concludes saying that multi-modality [at least two optima] exists in wing optimization. Also, strong coupling exists between the chord and thickness of the wing, which increases the multi-modality of the problem.

Further, the same author, \cite{Poole2}, refers to the constraint handling technique proposed by Deb \cite{Deb}. It also introduces the parallel decomposition of the niching DE algorithm tested on benchmark function and is reliable. However, these algorithms are expensive, as they require many CFD solutions. Some modification to the algorithm is required to reduce the cost function evaluation.

In the paper chernukhin \cite{oleg}, Gradient-Based multi-start (GB-MS) algorithm and hybrid algorithm are two algorithms adopted to optimize the wing surface. Implementing a system of linear geometric constraint will prevent infeasible geometries from being generated in GA \cite{oleg:phd}. The author confirms that 2-D airfoil optimization is unimodal, whereas 3-D wing optimization in subsonic flow conditions is multimodal. 

The author D.J Poole \cite{Poole3} introduced a total of 18 benchmark functions to test the DE-based niching method. Among them, nine are from literature, and remaining are from his work. Along with benchmark functions, different niching methods are also published. From results, niching methods based on \textbf{neighbourhood search} show better results in terms of quick convergence and reliability of stable niches. However, these algorithms require an extensive amount of cost function evaluations. Hence some modification is necessary to implement them on ASO.

\section{To concentrate upon}
This work basically focuses on using the Niching algorithms to obtain all possible optimal. The Niching techniques balance the population diversity by modifying selection, mutation, or crossover stages of Canonical DE. Since they construct on DE, the convergence rate is slow. When combined with these algorithms, the GB method will make them become fast, resulting in Hybrid optimization. The solution expected to converge faster. 

The scale factor $F$ and cross-factor $ CR $ have a significant impact on the algorithm's performance, such as the quality of the optimal value and convergence rate. There is still no right way to determine the parameters\cite{chen}.

According to chernukhin \cite{oleg}, surrogate modeling is challenging to handle when the number of design variables becomes large, which needs to be verified. 3D wing optimization with a viscous model not covered in the paper. However, this involves vast resources to optimize. With the available resources, it becomes difficult to handle. Little work has taken place in optimizing the wing with pre-defined winglet shape. The niching techniques involve a considerable quantity of cost function evaluations. So there needs to reduce cost function evaluation by modifying the code.

Few authors have solved this problem using the Genetic Algorithm-which evolve over time. The algorithm's inherent property is to get single optimal shape on each run, which leads us to think about the algorithms which provide all optimal shapes in a single run. The DE based parallel niching algorithms provides a better solution to these class of problems. 

