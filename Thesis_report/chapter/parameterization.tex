\chapter{Parameterization}
\label{parameterization}
\section{Background}
Aerodynamics shape optimization (ASO) is the process that involves optimizing an aero-shape (wing, airfoil, fuselage, so on) for a given constraint. Here the objective function can be Drag, Lift, Aerodynamic efficiency, so on, obtained from the flow solver (SU2 in this work). Since the wing requires the tedious job of plotting the actual surface, there is the need for simplifying the wing surface representation (surrogate modeling), the parameterization scheme came into existence, which represents the aero-shape mathematically. There are multiple ways to parametrize the wing, which include mesh-points, B-splines, cubic splines, Free-Form deformation. Representing the wing using the parameterization method results in less complexity in modeling as there will be few control points. The entire wing is developed based on the control points position. 

A shape parameterization system typically involves the coupling of a CAD tool and a grid generator. Every time the shape design variables are changed, a new grid must get generated without human intervention. For complex problems and especially involving RANS models, fully automatic grid generation can be difficult or impossible. An alternative approach is to use a reference shape, usually the starting design, and deform this shape by various techniques. The subsequent sections cover the detailed approached on FFD. It is necessary to generate a grid for the reference shape, which is deformed whenever it is deformed, thus avoiding the need to regenerate it.

Modern discretization techniques for differential equations yield high dimensional simulation models, which require high computational effort for determining approximate solutions. Such settings can be parameter studies, interactive simulations, parameter identification problems, statistical investigations, to name a few. To address these problems, efficient techniques for dimensionality reduction are desirable. In addition to fast algorithms, error quantification is crucial too. Methods like Reduced Basis (RB) techniques for parametrized partial differential equations and Model Order Reduction (MOR) for parametrized dynamical systems are designed to address these problems. 

\section{Problem statement}
Before discussing the parameterization and construction of the wing, the problem statement and constraints will be discussed initially.

The objective is to minimize $Cd$, subjected to constraints as mentioned,
$$  
\begin{aligned} 
C_{L} = {0.2625} \implies \text{Coefficient of Lift} \\ 
{C_{M_{x}}-0.1069 \leq 0} \implies \text{Root bending moment}\\
%{V_{initial}-V \leq 0} \implies \text{Volume}\\ 
%{S} = S_{initial} \implies \text{Baseline } \\
{-3.0} \leq \alpha \leq 6.0 \implies \text{AoA} \\
% -3.12  \leq \alpha_{2} \leq 3.12 \implies \text{Twist}\\
0.1  \leq thickness \leq 0.15 \implies \text{Thickness at five section}\\
% -0.45  \leq \alpha_{i} \leq 0.45 \quad \forall i \in[8,12] \implies \text{$\Delta$z$_{qc}$ at five section}\\
% -1  \leq \alpha_{i} \leq 1 \quad \forall i \in[13,17] \implies \text{$\Delta$x$_{qc}$ at five section}\\
0.7  \leq chord \leq 1.4 \implies \text{Chord at five section}\\
{2.7 \leq semi span \leq 3.3} \implies \text{Semi span}\\ 
\end{aligned}
$$ 

The AoA and $C_L$ constraint are satisfied by the CFD solver (SU2). The root bending moment constraint is implemented into the python code using feasibility rules proposed by Deb and Saha \cite{Deb}. The shape constraints like thickness, chord, semi span of a wing are satisfied using the python code coupled with FFD control points displacement, which will be discussed in further sections.

\section{Free Form Deformation}
Free-form deformation is a process by which shape change can be made to geometry by manipulating the location of points which are related to the geometry. Soderberg and Parry first introduced this technique in 1986. It can be correlated to the Bezier curves of parameterization of given curves. Before going directly with wing parameterization, a short introduction on the FFD applied to 2D geometries will be presented.

As mentioned before, FFD manipulates the geometry by attaching it to bezier curves. And these curves are defined by the grid of control points defined as $u \in[0,1]$, $t\in[0,1]$.The Bernstein polynomials govern the control points' influence on the space within the grid. The effect of perturbation reduces with the distance between mesh points from the control point. However, it is observed that moving the control points within the grid will have less effects, as shown in fig \ref{ffd_effect}.

\begin{figure}
    \centering
    \includegraphics[scale=0.56]{figures/ffd5.png}
    \caption{FFD control point grid with the a(2,3) control point displaced. The solid lined grid show how mesh points are shifted according to the influence of the control points\cite{ffd_book}.}
    \label{ffd_effect}
\end{figure}

The FFD, based on the grid of control points, is expressed mathematically as
\begin{equation}
\mathbf{X}(u, t)=\sum_{i=0}^{n} \sum_{j=0}^{m} \mathbf{a}^{(i, j)} f_{i}(u) g_{j}(t)
\label{ffd_2d}
\end{equation}

where \textbf{X}(u, t) is the Cartesian coordinates of the new, deformed location of a point at (u, t). And, $f_i(u)g_j(t)$ represents the Bernstein polynomial, $a^{(j,i)}$ represent the control point being displaced. To make this work, define a shape in parametric form and apply equation \ref{ffd_2d}. However, care should be taken to inscribe the entire object, subjected to deformation, into the FFD box; else, only the inscribed part will be deformed.

Extending further, same concept can be implemented to 3D problems. Every object point \textbf{X} has (s,t,u) coordinates in the parallelpiped coordinate region\cite{soderberg}.

\begin{equation}
\mathbf{X}=\mathbf{X}_{0}+s \mathbf{S}+t \mathbf{T}+u \mathbf{U}
\label{ffd initial equation}
\end{equation}

Let $\textbf{P}_{ijk}$, i = 0, \dots, l, j = 0,\dots, m, k = 0,\dots, n are control points on the lattice. Equation \ref{ffd_3d} represents the mathematical form of 3D FFD box equation as shown,
\begin{equation}
\mathbf{X}(u, t, s)=\sum_{i=0}^{m} \sum_{j=0}^{n} \sum_{k=0}^{p} \mathbf{a}^{(i, j, k)} f_{i}^{m}(u) g_{j}^{n}(t) h_{k}^{p}(s)
\label{ffd_3d}
\end{equation}
where Bernstein's polynomial $f_{i}^{m}(u)$ is defined as,

\begin{equation}
f_{i}^{m}(u)=\frac{(m) !}{(i) !(m-i) !} u^{i}(1-u)^{m-i}
\label{bernstein_poly}
\end{equation}

The new position of a point $\textbf{X}_{def}$ is computed by:

\begin{equation*}
\mathbf{X}_{d e f}=\sum_{i=0}^{l}\left(\begin{array}{l}
l \\
i
\end{array}\right)(1-s)^{l-i} s^{i}\left[\sum_{j=0}^{m}\left(\begin{array}{c}
m \\
j
\end{array}\right)(1-t)^{m-j} t^{j}\left(\sum_{k=0}^{n}\left(\begin{array}{l}
n \\
k
\end{array}\right)(1-u)^{n-k} u^{k} \mathbf{P}_{i j k}\right)\right]\end{equation*}

As compared to 2D FFD equation, 3D FFD requires an additional set of Bernstein's polynomial and a 3D grid points. Figure \ref{sphere_ffd} illustrates the influence of the control point movement on the sphere, when one of the control points is displaced away. Similarly, another example mentioned in figure \ref{wing_ffd} represents the wing shape deformation (preciously surface mesh deformation) due to wingtip control points being rotated and translated.

\begin{figure}
\parbox{0.49\linewidth}
{
\centering
 \framebox{ \includegraphics[width = 65mm,height = 65mm]{figures/sphere_ffd.png} }
 \caption{Influence of the control point over the sphere body with $a^{(m,n,p)}$ as $2 \times 2 \times 2$ control points.}
 \label{sphere_ffd}
}
\parbox{0.47\linewidth}
{
\centering
  \framebox{  \includegraphics[width = 65mm,height = 65mm]{figures/ffd_wing.png}}
  \caption{FFD of a wing shape, with the control points on the face of the wing-tip end are rotated and translated.}
  \label{wing_ffd}
}
\end{figure}

In figure \ref{wing_ffd}, the wing surface mesh is parameterized using the FFD with 27 control points. However, this work contains 60 control points to deform a wing surface mesh. Full details about the distribution of control points will be explained in the upcoming chapter.

\section{FFD implementation algorithm}
Evaluation and implementation of the FFD equation are explained in the pseudo code mentioned below. Input to this algorithm will be parametric coordinates of object surface mesh. Output of the algorithm will be the cartesian coordinates of deformed surface mesh.
\begin{algorithm}
\SetAlgoNoLine
\textbf{Input}: parameters (s,t,u)\\
\textbf{Output}: new coordinates of the point \textbf{X} for parameters (s,t,u)\\
  \For{i=0, \dots , l}
  {
    \For{j=0, \dots , m}
    {
        \For{k=0, \dots , n}
        {
        $X=X+\operatorname{binom}(l, i) * \operatorname{pow}(1-s, l-i) * \operatorname{pow}(s, i) *$
$\operatorname{binom}(m-1, j) * \operatorname{pow}(1-t, m-j) * \operatorname{pow}(t, j) *$
$\operatorname{binom}(n, k) * \operatorname{pow}(1-u, n-k) * \operatorname{pow}(u, k) * \boldsymbol{P}_{i j k}$
        }
    }
  }
 \caption{FFD algorithm}
 \label{ffd_algorithm}
\end{algorithm}

Note that function $binom$ is the computation of binomial number and function $pow(a, b)$ is power $a^b$.

\section{Reduced-order modeling}
After considering the effort involved in building the FFD box, the next issue to address is the algorithm's credibility over solving the higher dimension problem. At present, the problem's dimension is 125. The DE based niching algorithm is not tested for the 125-D dimension problem, which results in reconsider to reduce the problem's dimension. The preferred way to carry out the dimension reduction is Reduced-order modeling.

Complex geometries encountered in aerodynamics typically requires a large number of grid points to define the surface boundaries. For example, the representation of the standard wing involves a million+ grid points. Apart from this, if the number of random input variables (geometry variables) is not restricted, then the computational cost would increase linearly. To reduce the additional computational cost, reduced-order modeling of the geometry is necessary. Once the problem's dimension reduces into a manageable level, optimization of the NACA0012 wing can be implemented.

Reduced-order modeling is a highly complex topic and has found many applications in various fields of engineering. Typically reduced-order modeling methods can be categorized as parametric and non-parametric.

Parametric methods require a priori knowledge about the system to be reduced, whereas the non-parametric does not require any prior knowledge of the geometric uncertainty. One of the most important methods is Principal Component Analysis (PCA). PCA is also called by different names like Proper Orthogonal Diagnolisatopn (POD), the Hotelling transform, and the discrete Karhunen-Loeve transform (KLT).

PCA identifies the mutually uncorrelated basis vectors in the diminishing order of their importance. Also, it uses only the second-order statistical information (variance and covariance). All the higher-order information is discarded, resulting in a computationally efficient analysis.

In work published by Garzon \cite{garzon}, the successful reduced-order model can be derived using PCA for real fan blade measurements without any prior knowledge of the sources causing variations in the data. Furthermore, he reported using only five leading eigenmodes to capture 99\% of the scatter energy-measure of geometric variability.

The next section explains the mathematical formulation of PCA for the geometrical measurement data.

\section{Principal Component Analysis}
The underlying assumption behind this method is: the direction with the highest variance represents the features of interest and is achieved by diagonalizing the covariance matrix of the measurement data. Furthermore, this direction representing the highest variance is called principal components. A reduced-order model is obtained by selecting the first few principal components of the data.

Let the nominal surface geometry be defined by p coordinated points $x_{i}^{0}$ $\in$ $\mathbb{R}^{m}$, $i=1, \ldots, p$,  where m represents the data's dimension. Also, let n sets of measurements 
$\left\{\hat{x}_{i, j} \in \mathbb{R}^{m} \mid i=1, \ldots, p\right\} ; j=1, \ldots, n$ be available. Here index $j$ refers to a specific instance of measurement, and index $i$ refers to measurements representing a unique nominal coordinate point.

The error from the nominal geometry is given as,
$$x_{i, j}^{\prime}=\hat{x}_{i, j}-x_{i}^{0}$$

Further, subtracting the error vectors from their mean gives a centered

set of m-dimensional vectors,

\begin{equation}\tilde{x}_{i, j}=x_{i, j}^{\prime}-\bar{x}_{i} \mid i=1, \ldots, p ; j=1, \ldots, n\end{equation}

where the mean of these error vectors is calculated as,

\begin{equation}
\bar{x}_{i}=\frac{1}{n} \sum_{j=1}^{n} x_{i, j}^{\prime}, i=1, \ldots, p
\end{equation}

Representing the centered error vectors in a matrix form $\tilde{\textbf{X}}$ with $j^{th}$ column as, $\tilde{X}_{j}=\left[\tilde{x}_{1, j}^{T}, \ldots, \tilde{x}_{p, j}^{T}\right]^{T}$. This implies $\tilde{X}_{j}$ contains the centered set of error vector representing the $j^{th}$ set of measurement. $\tilde{\textbf{X}}$ will be generally of size $mp \times n$ matrix. The covariance matrix of $\tilde{\textbf{X}}$ is given by 
\begin{equation}
C_{x}=\frac{1}{n-1} \tilde{X} \tilde{X}^{T}
\end{equation}
As mentioned earlier, PCA typically identifies the directions that are mutually uncorrelated to each other. Along with this, it identifies the directions of the greatest variance of the data. This can be achieved by finding a transformation matrix which diagonalizes the covariance matrix $C_x$. One such method is obtaining the eigenvalue of the covariance matrix $C_x$. The more prominent way of approaching this problem is to perform a Singular Value Decomposition (SVD) of a matrix $\tilde{\textbf{X}}$\cite{ghate}.
\begin{equation}
\tilde{X}=M \Sigma N^{T}
\end{equation}

Here, M, N are $mp \times mp$, $n \times n$ orthonormal matrix respectively. And, $\Sigma$ is $mp \times n$ is a diagonal entries ordered in decreasing value. 

Reduced-order modeling is based on selecting the leading modes of the principal components obtained by finding the SVD of the covariance matrix $\tilde{\textbf{X}}$. However, to select a specific number of principal components, the scatter energy method is used. The total scatter energy $E$ is given by,
\begin{equation}\begin{aligned}
E &=\operatorname{tr}\left(\tilde{X} \tilde{X}^{T}\right) \\
&=\operatorname{tr}\left(M \Sigma N^{T} N \Sigma M^{T}\right) \\
&=\operatorname{tr}\left(M \Sigma \Sigma M^{T}\right) \\
&=\|M \Sigma\|_{F}^{2}
\end{aligned}
\label{random_energy}
\end{equation}
where $\|.\|_{F}^{2}$ is the Frobenius norm. Since the Frobenius norm is invariant under unitary multiplication,
$$E=\|M \Sigma\|_{F}^{2}=\|\Sigma\|_{F}^{2}=\operatorname{tr}\left(\Sigma \Sigma^{T}\right)=\sum_{i=1}^{m p} \lambda_{i}^{2}$$

where $\lambda_i$ is the $i^{th}$ diagonal entry of $\Sigma$. Selecting the first few modes from the matrix $M$ and taking note of their corresponding eigenvalues makes it possible to predict the amount of scatter energy being captured. This is a principal of using PCA based reduced-order modeling.

Another reason for using PCA based reduced-order modeling is the ease of implementation and computational efficiency of the PCA algorithm. SVD is the most popular way of implementing PCA. Predefined modules are available in both MATLAB\textsuperscript{\textregistered} and Python, which further strengthens the use of SVD.

Discussing the derivation of PCA based reduced-order modeling is out of scope here. Therefore, a reduced-order model for the geometric uncertainty can be written as,

\begin{equation}
x_{r}=x^{0}+\bar{x}+\sum_{k=1}^{n_{r}} a_{k} \mathcal{N}(0,1)
\end{equation}

where, $n_r \leq mp$ is the number of leading modes selected and $a^k$ is the $k^{th}$ column of $\frac{1}{\sqrt{n}}M\Sigma$. $N(0,1)$ is the random variable with zero mean and unit variance. As $n_r$ increases the total scatter of $x_r$ tends to $\tilde{x}$. Upcoming chapter deals in detail with the application of this method.

\section{Conclusion}
Understanding the concepts of FFD and PCA appears to be complicated. However, when it comes to implementation, both the methods are straight forward. For PCA, there are standard modules available in python. Implementing the FFD concept to a specific problem is a bit intricate. However, many articles are available to assist the user. The decision to select the number of control points in the FFD box plays a crucial role in obtaining various shapes of wings. The fewer number of control points will result in less significant wing shapes. At the same time, higher control points will result in a higher-dimensional problem, and the algorithm is not tested for higher-dimensional problems. A clever decision over the number of control points needs to be selected by considering these points—following which PCA is implemented. Again, based on the total scatter energy $ E $, a decision to select design variables is made. Generally, for aerospace applications, $E$ of more than 85\% is considered. Based on this (leading principal modes), the corresponding decision variables are selected. A detailed explanation about the FFD and PCA implementation to the problem of interest will be explained in chapter \ref{methodology}.