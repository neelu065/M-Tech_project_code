The thesis presented here aims at obtaining multiple optimal shapes for the NACA0012 wing. Beforehand, the Differential Evolution (DE) algorithm tested on the test function like Ackley function, Hypersphere, Egg-holder function, to name a few and found that the python optimization code implemented correctly. Using sophisticated DE-based niching algorithms like fDE, fNRAND1, fINRAND1, fCDE to name a few, it is possible to obtain local optima for test functions mentioned above.\\[2mm]

The DE based niching algorithms (with little or no modification) are further expanded to NACA0012 wing to obtain multiple local optima. The mesh points spread over the NACA0012 airfoil using a cosine function. A Free-Form Deformation (FFD) box with 60 control points is set-up over the NACA0012 wing, which resulted in 125 Dimensional (125-D) optimization problem. With the help of Principal Component Analysis, the problem dimension gets reduced to 10 based on its percentage of random energy ($\lambda^2$). A glyph script creates the wing tip followed by the volume mesh to all perturbed wings, and the SU2 solver will evaluate for Coefficient of lift (Cl) and Coefficient of drag (Cd).\\[2mm]

The entire optimization problem carried out in subsonic, inviscid condition (M = 0.4) subjected to several constraints with reasonably higher residual value. A full-fledged optimization code written in python and submitted to the HPC cluster via SLURM has resulted in two optima. A further modification to the existing design space results in additional optimal wing shapes.\\[4mm]

Keywords: Optimization, Parameterization, Niching algorithms, Differential Evolution, Free - Form Deformation, Principal Component Analysis