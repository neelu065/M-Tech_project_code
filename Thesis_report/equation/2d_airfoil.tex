Equation \ref{2D_airfoil_thickness} represents the formula to obtain the half-thickness of symmetric NACA four-digits airfoil.
\begin{equation}
y_{t}=5 t\left[0.2969 \sqrt{x}-0.1260 x-0.3516 x^{2}+0.2843 x^{3}-0.1015 x^{4}\right]
\label{2D_airfoil_thickness}
\end{equation}
where,\\
$x$ is position along the chord from 0 to 1.\\
$y_t$ is the half thickness for a given $x$.\\
$t$ is the maximum thickness represented in terms of chord.

Equation \ref{2D_airfoil_thickness} will result in the blunt airfoil at the trailing edge. If a sharp trailing edge is required, one of the coefficients should be modified to zero their sum. Modifying the last coefficient to $-0.1036$ will result in a small change in the overall shape of the airfoil, and a sharp trailing edge is obtained. 

Let ($x_U$, $y_U$), and ($x_L$, $y_L$) presents the upper and the lower airfoil surface coordinates respectively. Since the airfoil is symmetric, $y_c$ is equal zero.
This results in the airfoil coordinates as shown in equation \ref{2d_final_points}.
\begin{equation}
\begin{array}{ll}
x_{U}=x & y_{U}= +y_{t} \\
x_{L}=x & y_{L}= -y_{t}
\end{array}
\label{2d_final_points}
\end{equation}

