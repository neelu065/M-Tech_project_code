
% Define block styles

\tikzstyle{block} = [rectangle, draw, fill=blue!20, text centered, rounded corners, text width=10em]
\tikzstyle{line} = [draw, -latex']
\tikzstyle{block1} = [rectangle, draw, fill=blue!20, text centered, text width=10em]
\tikzstyle{block12} = [trapezium, trapezium left angle=70, trapezium right angle=110, text centered, draw=black, fill=blue!20, text width=10em]

\begin{figure}
  \centering
  \begin{tikzpicture}[node distance = 2cm, auto]
    % nodes
    \node [block] (continue) {\Large Continued ...};
    \node [block12, below of = continue] (parameter) {\large Parametric baseline wing};
    \node [block12, right of = parameter, node distance = 7cm] (cp_new) {\large Perturbed control points};
    \node [block1, below of= parameter, node distance = 2cm] (ffd_box) {\large FFD box equation};
    \node [block, below of= ffd_box, node distance = 2cm] (final) {\large Perturbed cartesian wing coordinates};
    
    % edges
    \path [line] (continue) -- (parameter);
    \path [line] (continue) -| (cp_new);
    \path [line] (parameter) -- (ffd_box);
    \path [line] (cp_new) |- (ffd_box);
    \path [line] (ffd_box) -- (final);
  \end{tikzpicture}
  \caption{Block diagram representing the input and output to FFD box equation.}
  \label{ffd_diagram}
\end{figure}