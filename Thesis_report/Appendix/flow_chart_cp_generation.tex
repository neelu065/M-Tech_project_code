
% Define block styles

\tikzstyle{block} = [rectangle, draw, fill=blue!20, 
text width=25em, text centered, rounded corners, minimum height=3em]
\tikzstyle{line} = [draw, -latex']
\tikzstyle{block1} = [rectangle, draw, fill=blue!20, 
text width=25em, text centered, minimum height=3em]

\begin{figure}
  \centering
  \begin{tikzpicture}[node distance = 2.5cm, auto]
    % nodes
    \node [block] (init) {\Large NACA equation with cosine function};
    \node [block1, below of = init] (new1) {\Large NACA airfoil (*.x3d)};
    \node [block1, below of = new1] (inte) {\Large Wing surface generation (baseline wing)};
    \node [block1, below of = inte] (inte2) {\Large FFD box over the baseline wing};
    \node [block1, below of = inte2] (inte3) {\Large FFD box corner control points};
    \node [block1, below of = inte3] (inte4) {\Large Interpolation of the FFD corner points};
    \node [block1, below of = inte4] (final) {\Large Cartesian coordinates of all control points (*.csv)};
    \node [block, below of = final] (continue) {\Large Continue ...};
   
    % edges
    \path [line] (init) -- node {\Large python script} (new1);
    \path [line] (new1) -- node {\Large extrude in span direction} (inte);
    \path [line] (inte) -- node {\Large glyph script} (inte2);
    \path [line] (inte2) -- node {\Large outcome } (inte3) ;
    \path [line] (inte3) -- node {\Large python script} (inte4) ;
    \path [line] (inte4) -- node {\Large outcome} (final) ;
    \path [line] (final) -- (continue) ;
  \end{tikzpicture}
  \caption{Flowchart representing the steps involved in generating the perturbed control points}
  \label{cp_flowchart}
\end{figure}
